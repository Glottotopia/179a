\chapter{Testing probability of origin}\label{ch:4}

\section{Introduction} \label{4.1}
This chapter is a demonstration of the statistical analysis of the degree of correspondence between the \textsc{sed45} and the Sranan45. It is a presentation of what this degree of correspondence says about the extent to which a \isi{mono-dialectal account} of origin is probable. In \chapref{ch:3}, I presented the formula with which I calculated this \isi{probability} as \textsc{p} = \textsc{ec}\slash\textsc{pc}. This meant that if I found a single \textsc{sed45} locality of 100\% correspondence with the Sranan45, the (\textsc{p})robability of this being a by chance occurrence would be calculated as the existing combinations (\textsc{ec}) of vectors divided by the total number of possible combinations (\textsc{pc}) of vectors in the \textsc{sed45}.

The above-mentioned formula would have been relevant if any (\textsc{ec}) vector in the \textsc{sed45} provided a 100\% match with the Sranan45 vector. However, as mentioned in \chapref{ch:3}, what the data presented, were partial matches. These partial matches ranged from a high of 60\% (\LSfrac{27/45}) to a low of 20\% (\LSfrac{9/45}). I therefore needed to revert to the secondary formula discussed in \chapref{ch:3} (see \sectref{3.4.3.2}), which could be used to calculate the statistical significance of these matches and by extension the \isi{probability} of these partial matches, in particular the (\LSfrac{27/45}) match, being a by chance occurrence. Before delving into the discussion of these partial matches, let me first discuss how the old \isi{probability} formula, which deals with total matches, would have worked.

\subsection{Calculating the possible combinations figure (\textsc{pc})} \label{4.1.1}
For any dataset containing horizontally ordered sequences of variants of the \textsc{sed45} word variables, there are two possibilities:

\begin{enumerate}
\item{There is specific combination of eight \isi{linguistic features} (see \sectref{3.3.2} in \chapref{ch:3}) across the 45 items in the \textsc{sed45}. In this case we can say that a ``match'' with the items in the Sranan45 has occurred. To illustrate this, take the putative \textsc{sed45} sequence [ars] \emph{arse}, [hɒg] \emph{hog} and [kəʊld] \emph{cold}, produced in Earl's Croome (Worcestershire) and compare it with the sequence of Sranan45  reflexes [ras], [hagu] and [kouru]. The words [ars] and [ras] match with +\textsc{pvr}, [hɒg] and [hagu] match with +h and +LexVar and [kəʊld] and [kouru] both match with the +Diph.}
\item{There is a different combination, in which case we can say that a ``non-match'' has occurred. Using the same hypothetical sequence above, a non-match would occur if the \textsc{sed45}  sequence was [aːs], [hɒg] and [kəʊld], as produced, for example, in Witton-le-Wear (Durham). This non-match would be with the variant [aːs] \emph{arse}, which needs to have the +\textsc{pvr} feature to be taken as the input \isi{etymon} for the Sranan45  reflex [ras].}
\end{enumerate}

Given these two possibilities, we can think of the \textsc{sed45} with their 45 \textsc{sed} items, as an ordered vector of length 45 \textsc{sed} word variants, with each \textsc{sed} word variant being either [+match] or [\textminus{}match] with its corresponding Sranan45 reflex. In using this binary system of categorization, the number of possible distinct combinations is calculated as 2\textsuperscript{45} (see \ref{exTable 4.1}).

% \begin{table}
% \begin{tabular}{l}
% \lsptoprule 
% \midrule 
\ea
\label{exTable 4.1}
% \caption
{Possible combinations (\textsc{pc}) of the variants of the \textsc{sed45} Variables}:\\
\[2^{45} = 3.52 \ast 10^{13}\]
This equates to a little over 35 trillion distinct vectors (\textsc{pc}). \\  
\z
% \lspbottomrule 
% \end{tabular}
% \end{table}

\subsection{Calculating the existing combinations figure (\textsc{ec})} \label{4.1.2}
A count of the number of localities, out of the 313 surveyed in the \textsc{sed}, which in the \textsc{sed45} database exhibited distinct variant combinations across the \textsc{sed45} etyma, revealed that with the exception of two localities, every \textsc{sed} locality had its own distinct vector. The one case of two locations producing the same vector of 45 variants involved neighbouring locations Ullesthorpe and Carlton Curlieu, both in Leicestershire. This meant that, as it related to the actually Existing Combinations (\textsc{ec}), of the variants in the \textsc{sed45}, there are 312 distinct vectors (\textsc{ec}). I could now calculate the \textsc{p}robability of a full (\textsc{ec}) Sranan45 target vector correspondence being a by chance occurrence, given that there are 312 (\textsc{ec})s and over 35 trillion distinct vector combinations (\textsc{pc})s.

\subsection{Calculating probability (\textsc{p})} \label{4.1.3}
Having secured the (\textsc{ec}) value and the (\textsc{ec}) value, we would then apply the statistical procedure that establishes the level of \isi{probability} of the required Sranan45 combination corresponding by chance with an actual (\textsc{ec}) in the \textsc{sed45}. We calculate the following \isi{probability}. There are $N = 3.52 \ast 10^{13}$ \textsc{pc} vectors, and there are $k = 312$ \textsc{ec} vectors. In addition to these, there is the ``target vector'' that represents the precise \textsc{sed} input for the Sranan45 vector. The \isi{probability} of this ``target vector'' being one of the (\textsc{ec}) vectors is simply $k/N$, i.e. \textsc{ec}/\textsc{pc}. This \isi{probability} is therefore:

% \begin{table}
% \captionsetup{justification=centering}
% \begin{tabular}{lllllll}
% \lsptoprule 
% \midrule 
\ea
\label{exTable 4.2}
% \caption
{Calculating \isi{probability} of an \textsc{ec} Vector and Sranan45 vector match being a by-chance occurrence}:\\
\[\frac{312}{3.52 \ast 10^{13}}  =   8.86 \ast 10\textsuperscript{-12} \]
This is a \isi{probability} of 886 in 1 trillion.
\z
% \lspbottomrule 
% \end{tabular}
% \end{table}

\REF{exTable 4.2} illustrates that given the large number of possible combinations of the 45 word variables, it is highly unlikely that finding an (\textsc{ec}) vector that matched the ``target vector'', i.e. an \textsc{sed45} vector combination that exhibits a 100\% match with the Sranan45, would have been a by chance event. As mentioned above, however, no such (\textsc{ec}) vector exists in the  \textsc{sed45} database. What exist in the database are 312 (\textsc{ec}) vectors which display partial matches with the target vector, exhibited in the Sranan45 database. We will now go into the discussion of these partial matches and the statistical tool, R, which was used to calculate the statistical significance of getting these partial matches and whether this was by chance.

\section{Calculating probability (\textsc{p}) for partial matches}\label{4.2}
The \isi{probability} calculations of the partial matches presented hereafter were undertaken using the \isi{statistical analysis tool} R (see \sectref{3.4.3.2}). I inputted the \textsc{sed45} and Sranan45 vectors into this computing environment and asked R to determine the level of statistical significance, with Alpha level 5\% (see \sectref{4.2.1} for discussion of Alpha), of the partial matches between the (\textsc{ec}) vectors in the \textsc{sed45} database and the Sranan45. R's built-in pbinom function, which is used in the calculation of the partial matches, was then utilized. This pbinom function is useful for summing consecutive binomial probabilities \citep{Stats}, which the partial matches' data represented. Before going any further into the discussion of \isi{binomial probability}, let me explain the concept of
Alpha level.

\subsection{Alpha (type I error)}\label{4.2.1}
The following discussion of Alpha is based on the California State University's webpage presentation on \emph{``Tests for Significance''}, last modified on April 22, 1998 \citep{Stats}, and \citet{Cowles82} article \emph{``On the Origins of the .05 Level of Statistical Significance''}.
If a researcher thinks that there is a relationship between two variables under investigation but the evidence says otherwise, then s/he has committed a Type I error. The \isi{probability} of committing this type of error is referred to as ``Alpha''. Researchers will most often specify the Alpha level that they are willing to accept. In the social sciences and areas of \isi{statistics} the Alpha level of .05 is most often chosen. This equates to a willingness to accept a 5\% \isi{probability} of making a Type I error of assuming a relationship between two variables when no such relationship exists. I opted to also use an Alpha level of .05. I did so, on the premise that if the relationship between the two vectors, i.e. the target vector for the Sranan45 and the partial matching (\textsc{ec}) vectors was strong (not by chance) or not (by chance), then with a small sample size of 312 (\textsc{ec})s, an Alpha level of .05 would detect this.

In light of the above discussion I not only looked at the highest matching (\textsc{ec}) vector, exhibited by the lect of Blagdon, with its 60\% match. I included those localities whose (\textsc{ec}) vectors of partial matches ranged from \LSfrac{22/45} to \LSfrac{27/45}, i.e. \textsc{ec} vector matches $\geq 48.89\%$ (see \tabref{Table 4.3}). These lower end matches were included since they satisfied the Alpha level of .05 (see \sectref{4.2.2}).

\begin{table}
\begin{tabular}{ll}
\lsptoprule 
 \textbf{Localities} & \textbf{/45}\\
\midrule 
Stogursey (Somersetshire), Pulham St. Mary (Norfolk), Yoxford \\ (Suffolk), Netheravon (Wiltshire), Whitechurch Canonicorum  \\ (Dorsetshire), Earl's Croome \& Hartlebury (Worcestershire),\\ Latterbridge \& Gretton (Gloucestershire), Little Bentley and \\ Tillingham (Essex), Outwood and Walton-on-Hill (Surrey) & \textbf{22}\\
East Mersea, Doddinghurst and Canewdon (Essex) &  \textbf{23} \\  
Wedmore and Horsington (Somersetshire) & \textbf{24}\\
Whitwell \textsc{i.o.w}. (Hampshire) &  \textbf{25}\\
Blagdon (Somersetshire) &  \textbf{27} \\
\lspbottomrule 
\end{tabular}
\caption{\textsc{sed} Localities with $\geq$ \LSfrac{22/45} matches with the Sranan45 target vector}
\label{Table 4.3}
\end{table}

\subsection{Binomial probability}\label{4.2.2}
In \chapref{ch:3}, I used the analogy of flipping an imaginary coin. Let us take out this imaginary coin again and go through the steps I went through in working with the partial matches data. Each (\textsc{ec}) vector of 45 matches and non-matches was constructed as a series of 45 independent coin tosses. In so doing the experiment became a binomial experiment because:

\begin{enumerate}
\item {R was asked to autonomously toss the coin for each of 45 items, per 312 \textsc{sed} localities;}
\item{R gave either heads, i.e. [+match] (which is equal to p) with one of the items in the target vector of the Sranan45, or tails, i.e. [\textminus{}match] with one of the items in the target vector. This failure to get heads on a toss was calculated as ($1-p$), i.e. 1 minus the \isi{probability} of getting a [+match];}
\item{I had a constant \isi{probability}, i.e. 0.5 (50\%), of getting [+match] on each of the 45 independent tosses of the coin;}
\item{Each coin toss was autonomous, so getting tails, for example, did not affect whether I got another tails, or even heads on another spin of the coin.}
\end{enumerate}

On each toss of our coin, the \isi{probability} of a [+match] was equal to \textsc{p}. The total number of [+match]'s ($r$), i.e. 22, 23, ... 27, would therefore have the binomial distribution $\text{binom}(45, p)$, with the ``mean'' of the distribution equal to $45p$ (total tosses times the value of $p$) and ``variance'', i.e. how widely the degree of (\textsc{ec}) matches vary across the data, equal to $45p(1-p)$. I therefore needed the value of $p$, with which it would be straightforward to calculate the threshold (minimum $\rightarrow$ maximum) number of matches (t), i.e. 22 $\rightarrow$ 27, such that the \isi{probability} $p(r \geq t)$ that the number of matches ($r$), in a given (\textsc{ec}) vector exceeded the threshold number of matches ($t$), was small, i.e. $<0.05$ (less than 5\%). If this \textsc{p} value was less than the threshold ($t$) I could then say that the \isi{probability} of getting partial matches was in no way a by chance occurrence. If I found any locality with a \textsc{p} value $>.05$, then I would conclude that the partial matching of such a locality was a by chance occurrence.

In applying this to the data, let us say that when I asked the \textsc{sed45} database if an \textsc{sed} item yielded a match in two varieties, the \isi{probability} \textsc{p} of a match was equal to $X$ number. I already knew that there were 45 items in each of the 313 \textsc{sed} varieties and my knowing this could therefore have been compared to a situation in which my imaginary coin was biased, with $\textsc{p}(\text{heads}) = X$ number (a baseline number that I still needed to calculate); I tossed the coin 45 times. For each time the coin reached the ground, I got ($r$) number of (heads) matches across the 45 features, across the 45 tosses. Two questions could have been asked from this coin tossing exercise:

\begin{enumerate}
\item {What is $p(r = 22 \text{ matches})$, i.e. what is the \isi{probability} that the total number of matches is \LSfrac{22/45}?}
\item{What is $p(r \geq 22 \text{ matches})$ i.e. what is the \isi{probability} that the total number of matches is equal to or greater than \LSfrac{22/45}?}
\end{enumerate}

Question 1, i.e. $p(r = \text{exactly 22 matches})$, was not the most interesting to ask. However, getting an answer to Question 2 would prove to be more interesting. This is because on one hand, the event (r = 22) is contained within the event ($r\geq 22$), and on the other hand, the latter event, i.e. ($r \geq  22$), led me to ask and attempt to answer two other important and interrelated questions. These were as follows:

\begin{enumerate}
\item {What is a remarkably ``large'' number of \textsc{sed45} to Sranan45 matches? The answer is ``large'' = r > some number (t), i.e. threshold, such that the \isi{probability} of this ``large'' value occurring by chance is ``small'', i.e. $<0.05$ as discussed above.}
\item{What is the number of matches, such that $p(r \geq t)$ is $<0.05$? The \textsc{sed45} and Sranan45 data were inputted into R and the \textsc{sed45} data were asked the following two questions as it related to its degree of matching with the  Sranan45 target vector: ``What is the threshold number of matches (t), such that if an observed vector (\textsc{ec}) has more than (t) matches i.e. 22, 23, 24, ... 27, we could say that such an event is unlikely to have occurred by chance?'' In particular, ``is the observation of Blagdon's \LSfrac{27/45} match an ``unlikely'' event under the chance hypothesis, and strong evidence in favour of the hypothesis that the Blagdon dialect is the source of the target vector for the Sranan45 reflexes?'' Before R could calculate this request there was a need to construct an approximate model for the varying distribution of matches illustrated in \tabref{Table 4.3}.}
\end{enumerate}

A simple estimate of p was taken as the proportion of all vector elements that were [+match] between the \textsc{sed45} and the target vector, which is exhibited by the Sranan45 reflexes. Using the degree of matches across the data in the \textsc{sed45}, an estimate for p was calculated as follows:

\begin{enumerate}
\item {Start with the low proportion of vectors having \LSfrac{21/45} or fewer matches, which is equivalent to \LSfrac{292/313},  i.e. the non-matches (tails figure) of 292 (\textsc{ec})s out the actual 313 in the data.}
\item{Equate this \isi{probability} to the lower tails \isi{probability} of $b(45, p)$; this resulted in the equation below:}
\end{enumerate}


% % \begin{table}
% % \begin{tabular}{l}
% % \lsptoprule 
\ea\label{Table 4.4}Securing an estimate for \textsc{p}\\
\begin{lstlisting}
fp = functionP (pbinom(21, 45, p) - 292/313)^2
optimize (fp, c(0,1))
$minimum
[1] 0.367463
\end{lstlisting}
\z
% \lspbottomrule 
% \end{tabular}
% % % \end{table}

What the formula in \REF{Table 4.4} calculated was a value for p within the event (21 [+match]es, from out of 45 independent trials, minus the tails figure of \LSfrac{292/313}, remaining (non-matching) vectors, all squared), where the conditions, based on the structure of the data, are that there is 0.5 chance of getting a 0, i.e. [\textminus{}match], and a 0.5 chance of getting a 1, i.e. [+match]. The result, which is equivalent to the minimum threshold value between $22\rightarrow27$, was $p = 0.367463$. R's optimize function was used to do this since it yields the value that minimizes a function over a specified interval, i.e. $22\rightarrow27$ in this case. The resulting estimate for \textsc{p} made it possible to ask R to calculate p(r $\geq$ t), for $t = 22, 23, ..., 27$. This was done as follows:

% % % \begin{table}
% % % \begin{tabular}{l}
% % % \lsptoprule 
\ea Calculating $p(r \geq t)$ (see \ref{Table 4.4} for the calculation for {\ttfamily p0})\\\label{Table 4.5}
\begin{lstlisting}
$Objective # (this is the value of functionP at maximum)
$[1] 2.800027^-10 
p0 = 0.367463 
1 - pbinom(c (21:26), 45, p0) 
# which is: 
1 - pbinom(c (21:26), 45, 0.367463)
\end{lstlisting}
\z
% % % \lspbottomrule 
% % % \end{tabular}
% % % \caption{}
% % % \label{Table 4.5}
% % \end{table}

The resulting probabilities of the calculations illustrated in \REF{Table 4.5}, i.e. $p(r\geq t)$ for $t = 22, 23, ..., 27$, were as follows:

\begin{table}
\begin{tabular}{cS[table-format=1.4]}
\lsptoprule 
{($t$) Threshold}  & \multicolumn{1}{c}{Probabilities} \\
\midrule 
{22} & 0.0641  \\  
{23} &  0.0343 \\ 
{24} &  0.0170   \\
{25} & 0.0078\\
{26} & 0.0033 \\
{27} & 0.0013 \\
\lspbottomrule 
\end{tabular}
\caption{Results of the calculation for $p(r \geq t)$}
\label{Table 4.6}
\end{table}

These calculations, using the approximate model, suggested that \LSfrac{23/45} matches or more might be regarded as unusually high at the 5\% significance level. This corresponded to vectors with more than 50\% matches. Based on the calculations presented in \tabref{Table 4.6}, the list of localities presented in \tabref{Table 4.3} was reduced to only those \textsc{sed} localities whose (\textsc{ec}) to Sranan45 correspondence was unusually high (statistically significant) at the 5\% significance level. The following localities were removed from the list: Stogursey (Somersetshire), Pulham St. Mary (Norfolk), Yoxford (Suffolk), Netheravon (Wiltshire), Whitechurch Canonicorum (Dorsetshire), Earl's Croome and Hartlebury (Worcestershire), Latterbridge and Gretton (Gloucestershire), Little Bentley and Tillingham (Essex), and Outwood and Walton-on-Hill (Surrey). All of them exhibited matches of \LSfrac{22/45}. Having excluded these thirteen localities, I was left with seven localities with (\textsc{ec}) vectors exhibiting $>50\%$ correspondence with the Sranan45 reflexes. These localities are presented in \tabref{Table 4.7}.

\begin{table}
\begin{tabular}{lll}
\lsptoprule 
\textbf{Localities} &\textbf{/45}& \textbf{\textsc{p}} \\
\midrule 
East Mersea, Doddinghurst and Canewdon (Essex) & 23 & \textbf{.0343}  \\  
Wedmore and Horsington (Somersetshire) & 24 & \textbf{.0170} \\ 
Whitwell \textsc{i.o.w} (Hampshire) & 25 & \textbf{.0078} \\
Blagdon (Somersetshire) & 27 & \textbf{.0013}\\
\lspbottomrule 
\end{tabular}
\caption{\textsc{sed} Localities with > 50\% Matches with the Sranan45 Target Vector}
\label{Table 4.7}
\end{table}

\tabref{Table 4.7} highlights the fact that the Blagdon (\textsc{ec}) has an almost six times lower \isi{probability}, i.e. 0.0013, of its partial correspondence being ``by chance'' than does Whitwell Isle of White (Hampshire). Whitwell Isle of White, hereafter \textsc{i.o.w}., has the second lowest by chance score of 0.0078. Blagdon has an almost thirteen times lower \isi{probability} than the third set of localities, i.e. Wedmore and Horsington (Somersetshire), with their \isi{probability} of 0.0170. Blagdon also has about twenty-six times lower \isi{probability} of its partial matching being by chance than does the fourth set of localities; i.e. East Mersea, Doddinghurst and Canewdon (Essex) with their 0.0343 \isi{probability}. Consequently, Blagdon, with its 0.0013 \isi{probability} of its 60\% (\LSfrac{27/45}) match with the target vector not being by change, stood out as the most statistically likely source of the Sranan45 reflexes.

The high statistically significant probabilities exhibited by the six remaining localities could not be neglected however, since all six `not by chance' localities, i.e. Whitwell \textsc{i.o.w}., Wedmore, Horsington, East Mersea, Doddinghurst and Canewdon, could also be considered potential sources for the Sranan45. Consider two further facts concerning these remaining six localities:

\begin{itemize}
\item {three of these seven localities, i.e. East Mersea, Doddinghurst and Canewdon, are situated in the East and East Anglia region of \isi{England}, specifically in the same county of Essex;}
\item {the remaining three, i.e. Whitwell, \textsc{i.o.w}. (Hampshire), Horsington and Wedmore (Somersetshire), are alongside Blagdon, situated in the western part of the south of \isi{England}.}
\end{itemize}

One question came to mind when I noticed this pattern. This question was as follows: ``is there a linguistic relatedness between the south and east of \isi{England} and if so what does it tell us about the nature of the \ili{English} \isi{dialectal input} into \ili{Sranan}?'' Trying to decipher this tale of potential linguistic relatedness would involve my looking at the degree of linguistic overlap between these seven localities. The steps taken to answer the question of linguistic relatedness are discussed in detail in \chapref{ch:7}.

\section{Assessment of the results of the statistical analysis}
The results of the statistical analysis pinpointed a statistical significant lect -- Blagdon (Somersetshire), albeit with a partial match, i.e. \LSfrac{27/45}, with the Sranan45 target vector. However, though Blagdon was identified as being the most statistically significant putative \isi{input lect}, six other lects were presented as being of near statistical significance to Blagdon's 60\% correspondence with the target vector. These lects are situated to the east of \isi{England}, specifically in Essex in East Anglia and the south of \isi{England}, specifically in Somersetshire and Hampshire. These results presented me with two possibilities:

\begin{description}
\item[Possibility 1:] Blagdon, being the most statistically significant input was the \isi{input lect} for the Sranan45. Its remaining 40\% (\LSfrac{18/45}) non-correspondences with the Sranan45 might therefore be attributed to internal and external language change overtime. This possibility seemed worthy of serious consideration, since the \textsc{sed} was conducted over 300 years from the period of interest, which is from 1650 to 1667 (see \chapref{ch:6}).
\item[Problem:] I did not want to make any hasty conclusions, especially because possibility 1 fails to account for the remaining six localities whose partial matches could also be attributed to internal and external language changes over the 300-year period. This means that any one, if not all of the seven localities, i.e. Blagdon, Whitwell, Wedmore, Horsington, Canewdon, Doddinghurst and East Mersea, could have possibly been the input for the  Sranan45.
\item[Possibility 2:] The input for the Sranan45 was from two major regions in which there was a group of localities in close geographical proximity, with one of these localities forming the core of the \isi{linguistic influence}. This possibility seemed very feasible since I could essentially cluster the seven localities of statistical significance, according to region and the counties within these regions, in which each locality is located (see \chapref{ch:5}).
\item[Problem:] Possibility 2 seemed to be the more viable of the two possibilities. This was particularly because the seven localities formed two non-contiguous clusters; these being Blagdon, Horsington, Whitwell, Wedmore in the  south of \isi{England} and Canewdon, Doddinghurst and East Mersea in the east and East Anglia region. Still, as with possibility 1, I did not want to jump to conclusions.
\end{description}

Given the problems with the two possibilities presented above I wanted to assess the  \textsc{sed45}$\sim$Sranan45 correspondence data anew, via an alternative approach, i.e. via the use of Linguistic Feature Mapping. I wanted to see if in using this alternative approach, I would arrive at results that would corroborate, add to, or disconfirm the results of the statistical analysis that were presented in this chapter.
