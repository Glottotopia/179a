\addchap{Preface}
\begin{refsection}

The fundamental aim of this research is two-fold. First, it represents an attempt to pinpoint the precise origin of the early \ili{English} lexical and phonetic (lexico-phonetic) influences in \ili{Sranan}; i.e. whether this influence was from a single dialect, as expressed by a \isi{mono-dialectal account} of origin, or from a composite of \isi{dialects} from all over \isi{England}, as expressed by a \isi{pan-dialectal account}. Second, it introduces a new methodological tool (comprising of a \isi{statistics} component, an \ili{English} \isi{dialect geography} component and a 17\textsuperscript{th} century \ili{English} migration history component) with which such linguistic reconstructive work can be done. This tool was used to ascertain the potential dialectal origins of forty-five \ili{Sranan} words of \ili{English} origin, alongside the \isi{dialectal origin}(s) of their speakers. This was done via corroboration of the results of the independent analyses done across the three components of the tripartite methodological tool. The work relies heavily on secondary data sources for both the \ili{Sranan} data and \ili{English} dialectal data. The reason for this is the need to use the oldest possible lexical and phonetic information for both language varieties since the early 17\textsuperscript{th} century \ili{English} influence in \isi{Suriname}, the country in which \ili{Sranan} is spoken, ended after 1667.
\printbibliography[heading=subbibliography]
\end{refsection}

