\chapter{Brief overview: Views on superstrate influence}\label{ch:2}

%Section2.1 begin %%%%%%%%%%%%%%%%%%%%%%%%%%%%%%%%%%%%%%%%%%%%%%
 \section {``{Creole}'': The people} \label{2.1}\il{Creole}
 An online search for additional literature, to add to those already sourced for this research, concerning \isi{superstrate} and creole \isi{languages} respectively, led to an article entitled \textit{The Context of Wide Sargasso Sea > Social\slash Political Context > \ili{Creole} Identity and Language}.\footnote{\url{http://crossref-it.info/textguide/Wide-Sargasso-Sea/29/1915}, last accessed on November 12, 2018.} This article presented a discussion about the social and political context in relation to \ili{Creole} identity and language in the novel Wide Sargasso Sea. The novel, written by Dominican-born author Jean Rhys, which is set in the 1966 postcolonial era, told the story of heiress Antoinette Cosway. The article mentioned the fact that the term ``\ili{Creole}'' was originally used to refer to \isi{Whites} of \ili{British} (like Rhys' Antoinette) and/or of other European parentage, who were born in the \isi{Caribbean}. A subsequent read of \citegen{Page97}  review of Wide Sargasso Sea highlighted the following:
 
  \begin{quotation}
[Antoinette] ... is descended from the plantation owners, and her father has had many children by Negro women. She can be accepted neither by the Negro community nor by the representatives of the colonial centre. \underline {As a white creole she is nothing}. The taint of racial impurity, coupled with the suspicion that she is mentally imbalanced brings about her inevitable downfall .... 
\end{quotation}

Confronted, again, by the concept of ``White \ili{Creole}'', there was a need to find out more about this concept and this led to the works of \citet{Saxon89} and \citet{Ward04}. Both works expressed the fact that in New Orleans, for example, the term ``\ili{Creole}'' provoked two distinct responses. The first response was that \isi{Creoles} were children of  ``... European parents born in a \ili{French} or \ili{Spanish} colony'' \citep[270]{Saxon89}. The second response was that the term refers to ``someone whose ancestors came to the colonies from France or Spain, who was born in Louisiana, and who may be light-skinned and black as well'' \citep[xiv]{Ward04}. However, the \isi{Creoles} who gave the first response, would never agree with the last half of the second response, i.e. that \isi{Creoles} can be ``light-skinned and black''. This is despite the fact that their own ancestors ``... fathered mixed-race children'' with Black slave paramours \citep[xiv]{Ward04}.

The term ``\ili{Creole}'', when it is applied to \isi{Whites}, such as Rhys's Antoinette, most often refers to rich, upper class, and land-owning descendants of Europeans, who were born in the Americas (see \citealt{Cassidy82}). These White \isi{Creoles}, by virtue of the fact that their nannies, playmates and servants were most often enslaved Africans, spoke a creole language such as \ili{Sranan} \citep{Cassidy82}. However, the main focus of this research is not on this group of people. Instead, the focus is on the speech of poor \isi{Whites}, these being the indentured servants who came to the \isi{Caribbean} colonies. These poor \isi{Whites} and their descendants, who came to exist in socially isolated communities, preserved to varying extents the language forms of their \ili{British} forefathers (see \citealt{Aceto10} and \citealt{Williams03}).

Apart from the use of the term ``\ili{Creole}'' in reference to ``pure'' (unmixed) \isi{Whites} of \isi{Caribbean} origin, the line ``... As a white creole she is nothing ...'' stood out to me. It made me wonder about the language of these White \isi{Creoles}, and also of the poor \isi{Whites}, who in large numbers migrated to the colonies as indentured servants (see \chapref{ch:6}). According to historians such as \citet{Galenson02}, it was common practice for enslaved Africans and white indentured servants to work side-by-side on plantations. This sociolinguistic situation represented a perfect scenario for intense \isi{language contact}. Black speech in \isi{Caribbean} had been widely studied but had \ili{Creole} linguists ever thought to study the language of these poor \isi{Whites}? The answer to this question was yes, but not to any significant degree. According to \citet{Williams03}, early research of this kind had been ``hindered by a lack of knowledge of ``white'' \isi{dialects} ... [and] ... the work that has been done has generally suffered from a lack of understanding of the social and cultural dynamics of ``whiteness'' in Anglophone West Indian contexts'' \citep[95]{Williams03}.

How would the findings of this research relate to the body of research on the language of the descendants of these poor \isi{Whites}? This was one of three important questions to address whilst tackling the three research questions presented in \chapref{ch:1} (see \chapref{ch:7} for the remaining two questions). In \chapref{ch:7}, I compare the findings from the contemporary works of \citet{Aceto10}, \citet{Williams03}, \citet{Blake04} and others, and the significance of their work within the context of my findings. These linguists researched what \citet[30]{Trudgill02} referred to as the ``lesser-known varieties of \ili{English}''. These are a set of relatively unstudied, native varieties of \ili{English} that are spoken in some parts of the \ili{English}-speaking world \citep{Trudgill02}. What is the language spoken by the descendants of these poor \isi{Whites}? Is it \ili{English}, or a ``Caribbeanized'' dialect of \ili{English}? In most cases, it is the same or similar to the linguistic code used by \isi{Blacks} (see \citealt{Cassidy82}).

 %Section2.1 ends %%%%%%%%%%%%%%%%%%%%%%%%%%%%%%%%%%%%%%%%%%%%%%
 
 %Section2.2 starts %%%%%%%%%%%%%%%%%%%%%%%%%%%%%%%%%%%%%%%%%%%%%%
 \section {``{Creole}'' language and theories of genesis}\label{2.2}\il{Creole}
What exactly are ``creole'' \isi{languages} and how are they formed? McWhorter, in his work \emph {Defining Creole} \citep{McWhorter05}, expressed the fact that after years of established \ili{Creole} Language Studies, no consensus had yet been reached as to what a creole really is. Defining this concept, according to  \citet{McWhorter05}, was/is such a sensitive issue that some researchers felt it best to leave it alone. Notwithstanding the challenge involved in defining the term ``creole'', various linguists, such as \citet{Kihm80} and \citet{Chaudenson92}, used the word as a socio-historical term ``referring to certain \isi{languages} born as \isi{lingua} francas amidst heavy contact between two or more \isi{languages}'' \citep[9]{McWhorter05}. This definition is an example of one of three main types of definitions that \citet{Hickey97} claimed to have identified within the field of \ili{Creole} Linguistics. He classified these three types as ``External'', ``Acquisitional'' and ``Structural''.

An External definition, such as that presented above, considers factors outside of the language itself; a ``creole'' is therefore defined based on its sociolinguistic historical context of development. Acquisitional definitions identify creoles as language varieties that arose in situations where a generation of speakers developed these \isi{languages} from significantly reduced and imperfectly acquired colonial lexifiers. Structural definitions identify creoles as \isi{languages} that have undergone reformation with respect to their \isi{lexifier language}(s) and possibly substrate \isi{languages} \citep{Hickey97}.

These definitional types originated from the various theories of \isi{genesis} that had been presented for the various \isi{languages} labelled ``creoles''. These theories can be grouped together according to those that focus on the influence of European lexifiers, those that focus on non-European influence, and those that regard ``... universals of language acquisition and/or language-internal development as the crucial factor in \isi{creole genesis}'' \citep[3]{Braun09}. As with the three categories of definitions for ``creole'' highlighted by  \citet{Hickey97}, the theories of \isi{creole genesis} can be grouped into three main types. The more contemporary versions of these are briefly presented hereafter, under the headings \emph{Eurocentric theories of creole genesis}, \emph{Non-Eurocentric theories of creole genesis} and \emph{Language universal theories of creole genesis}. The presentation of these approaches is followed by a discussion of where all three types of approaches converge.

\subsection {Eurocentric theories of creole genesis} \label{2.2.1}
Theories that regard the impact of European influences as being the most vital to \isi{creole formation}, range for example, from the early works of \citet{Bloomfield33}, \citet{Hall66}, \citet{Ferguson71}, to the more recent works of \citet{Chaudenson92}, and \citet{Mufwene01, Mufwene08, Mufwene08b}. The core principle of Eurocentric theories of \isi{genesis} is that creoles are approximations of their European lexifiers \citep[43]{Baker00}.  \citet{Mufwene08}, for instance, suggested that ``in all ... cases of language evolution, the action of competition among competing ... systems is evident '' \citep[58]{Mufwene08}. Furthermore, in \isi{language contact} settings involving one target language (the lexifier), with the other language(s) involved offering some competition, ``... learners ... normally approximate the pattern provided by speakers of the target language'' \citep[122]{Mufwene08}. The same is claimed to be true for those socio-historical contact situations that resulted in the linguistic systems referred to as ``creoles''.

In these situations, i.e. those leading to the formation of ``creoles'', Europeans and non-Europeans interacted regularly and since non-Europeans had no one with whom to use their own ethnic language(s), the features of the founder populations, i.e. of the \isi{European language}, had the advantage in the majority of their linguistic forms being selected over non-European ones \citep{Mufwene96}. One effect of this was that children born into such situations, and also their parents, might not have seen knowledge of such non-European \isi{languages} as particularly necessary \citep{Mufwene08b}. In fact, this might be one of the major reasons why ``... structural features of creoles ... [are seemingly] ... predetermined to a large extent (though not exclusively) by characteristics of the vernaculars spoken by the populations that founded the colonies in which they developed'' \citep[28]{Mufwene96}. Mufwene's assertions, as with those of many other superstratists, were influenced by what \citep[3]{Braun09} referred to ``... as one of the much debated ... [\isi{Superstratist}] ... approaches'', i.e. Chaudenson's \citep{Chaudenson92} theory of \isi{genesis}.

According to  \citet{Braun09},  \citet{Chaudenson92} viewed creoles ``as modifications of non-standard European superstrates with little influence from the native \isi{languages} of the slaves'' \citep[3]{Braun09}. \citet{Chaudenson92} asserted that when we closely examine how colonial societies got their start, we will see that ``... the duration of the period during which \isi{Whites} were more numerous than \isi{Blacks} was considerable, and conditions did not change on the very day the black population outnumbered the white population...'' \citep[60]{Chaudenson92}. Since this was the case, what he referred to as `first generation creoles', such as \ili{Sranan}, \ili{Jamaican}, etc., would therefore be approximations of the \isi{European language}(s) spoken by the \isi{Whites}. This assertion was voiced from within his earlier work on \ili{French} creoles in which he posited that these creoles, i.e. \ili{French}-\isi{lexicon} creoles, could be traced back to a particular \ili{French} dialect spoken in 17\textsuperscript{th} and 18\textsuperscript{th} century France, i.e. the dialect that was in use in the Normandy region \citep{Chaudenson79}.  \citet{Chaudenson79} claimed that there was no need for \isi{Blacks} to outnumber the \isi{Whites} for a creole to have developed. In fact, (\ili{French}) creoles would have developed while the \isi{lexifier language} was still relatively accessible. In such socio-linguistic conditions, where the \isi{Whites}, who spoke vernacular \ili{French}, still outnumbered the \isi{Blacks}, these \isi{Blacks} were no doubt motivated to abandon their native language(s) to be able to communicate with others \citep{Chaudenson79}.

In defence of his assertions,  \citet{Chaudenson01} stated that even those theorists that advocate for a non-European focus in \isi{creole genesis} all agree that ``for the most part, creolization has occurred through collective, imperfect, and approximate ``learning'' of \ili{French} (or, more generally, any other lexifier) ...'' \citep{Chaudenson01}. The most extreme of these Non-Eurocentric theorists is Mervyn Alleyne, who in his \emph{Comparative Afro-American} work \citep{Alleyne80} claimed that the most essential language influences during the periods of creole geneses, specifically for \ili{English}-\isi{lexicon} creoles, were the Non-European (African) \isi{languages} present in the colonies. This is discussed in the following section.

\subsection {Non-Eurocentric theories of creole genesis} \label{2.2.2}
Non-Eurocentric accounts of origin, which are hereafter referred to simply as \isi{Substratist} approaches, have their origin in the early 19\textsuperscript{th} century works of philologists such as  \citet{Baissac80}. The essence of the \isi{Superstrate} approaches is that ``creole'' \isi{languages} owe much of their development to the influence of the African (substrate) language(s) that were present during the \isi{creole formation period}. \citet{Mufwene90, Mufwene96b} identified three main types of \isi{Substratist} models: the first type, for which the proponents are  \citet{Alleyne80, Alleyne96} and  \citet{Holm89}, ``identifies the source of individual features in diverse substrate \isi{languages}... that must have been represented in the ethnolinguistic ecological setting of the relevant contact'' \citep[167]{Mufwene96b}; the second type, endorsed by  \citet{Lefebvre98, Lefebvre04} and  \citet{Lumsden99}, posited a process of ``relexification'', i.e. the replacement of L1 (substrate language) lexical items with their L2 (European) counterparts; the third type of substratist approach, endorsed by  \citet{Keesing88}, ``... validates substrate influence with the relative typological homogeneity of \isi{languages} in contact with the lexifier'' \citep[167]{Mufwene96b}.

The substratist theory of \isi{creole genesis} presented by \citet{Alleyne80} could be considered to be the most extreme of those highlighted in the previous paragraph \citep{Byrne87}. In fact, his account is in stark contrast to the most extreme superstratist approach (see \sectref{2.2.1}).  \citet{Alleyne80} asserted that the creole \isi{languages} that he focused on (Atlantic \ili{English} \isi{lexicon} \isi{Creoles}), in which he included the \textsc{ssa} creoles, are essentially a continuity of the African \isi{languages} that went into their development. Alleyne supported his claim via the presentation of his observation that ``... the rules which account for [serial verb constructions] are basically the same in... [the Atlantic \isi{Creoles}] as in \ili{Kwa} \isi{languages} [African substrate]'' (\citeyear[167]{Alleyne80}) and that the characteristics of serialization in the \ili{Kwa} \isi{languages} ``seem to be closer to \ili{Saramaccan} than to other... [Atlantic \ili{Creole}]... \isi{dialects}'' \citep[167]{Alleyne80}. Alleyne based most of his assumptions on the observations he made of \ili{Saramaccan}, which in his view ``... may represent the oldest layer of creole known to us...'' and thus the least altered from its substrate \citep[91]{Alleyne79}.

The \textsc{ssa} creoles seemed to represent a haven for most creole linguistic study, including my own, given their peculiar differences from other \ili{English} creoles, such as [kouru] for \textit{cold} where others have [kuol]. These creoles were/are so favoured that even those theorists who advocate for linguistic universals, as opposed to (Non-)Eurocentric accounts, make reference to them, specifically \ili{Saramaccan}, as ``radical creoles'' (cf. \citealt{Bickerton77, Bickerton81, Bickerton84, Bickerton99}). These radical creoles as expressed by  \citet[3]{Byrne87} were so-called because ``[their] ... grammars come closer to approximating the unmarked state of our innate, genetically endowed facult\'{e} de langage.'' Just as the \isi{Superstratist} and \isi{Substratist} schools had/have their most extreme advocates in the form of \citet{Chaudenson01} and  \citet{Alleyne80}, respectively,  \citet{Bickerton77, Bickerton81, Bickerton84, Bickerton99} is the theorist most associated with Universalist theories of \isi{creole genesis}.

\subsection {Language universal theories of creole genesis} \label{2.2.3}
Bickerton's (\citeyear{Bickerton77, Bickerton81, Bickerton84, Bickerton99}) Language (Universal) Bioprogram Hypothesis, hereafter \textsc{lbh}, asserted that creole \isi{languages}, in particular \ili{Saramaccan}, closely approximated, neither their superstrates nor their substrates but the inborn, genetically endowed, ``facult\'{e} de langage'' that all humans are born with \citep[158]{Bickerton84}. The importance and reliance in this `faculty of language' is echoed in the work \citet{DeGraff01}, who defined ``creole'' as ``... the product of extraordinary external (sociohistorical) factors coupled with ordinary (internal) linguistic resources inherent to the human facult\'{e} de langage'' \citep[11]{DeGraff01}.

Bickerton's (\citeyear{Bickerton77, Bickerton81, Bickerton84, Bickerton99}) biological predisposition for language comes with certain preset features such as a natural tense-aspect schema that is embedded in certain neural pathways of the brain  \citep{Bickerton75}. In his own words,  \citet[2]{Bickerton81} stated the following: ``... all members of our species are born with a bioprogram for language which can function in the absence of adequate input.''

Bickerton's (\citeyear{Bickerton77, Bickerton81, Bickerton84, Bickerton99}) hypothesis seemingly paralleled particular components of \isi{Superstratist} theories. The reasons for making this assertion are found in the following aspects of Bickerton's (\citeyear{Bickerton77, Bickerton81, Bickerton84, Bickerton99}) theory of the \isi{genesis} of creole \isi{languages}, specifically \ili{English} creole \isi{languages}:

\begin{enumerate}
\item European plantation owners in need of a large-scale labour force imported enslaved Africans to work for them. Consequently, pidginization of these European \isi{languages} took place. This pidginization process was in essence second-language learning, characterized by limited access to, and therefore inadequate acquisition of, the European lexifier \isi{languages} \citep{Bickerton77}. This, of course, resulted in the demise of the African \isi{languages} which had little or no part to play in the process.
\item Children born into these multi-linguistic ecologies, whilst learning these relatively unstable pidgins \citep[49]{Bickerton77}  as their first \isi{languages}, ``[received] ... restrictive input ...'' and for this reason their  built-in grammars, which are conditioned by the bioprogram, were triggered, and continued into adulthood (cf. \citealt{Bickerton79, Bickerton84, Bickerton81}). Whilst all this was occurring, the adults' linguistic repertoires, with more and more access to the European lexifiers, were also developing. These adults were moving away from both their L1s (native \isi{languages}) and a soon-to-be obsolete and extremely flexible pidgin \citep{Bickerton81}. \ili{Creole} \isi{languages} were the linguistic outcome of these two co-occurring events.
\end{enumerate}

The \textsc{lbh}, however, disregarded the creole \isi{languages} of the descendants of ``White \isi{Creoles}'' and poor White indentured servants (see \citealt{Schumann83, Cassidy82}), which were/are spoken in \isi{Suriname} and around the \ili{English} world (see \citealt{Trudgill02}). 

\subsection {Where all theories of {creole genesis} concur}\label{2.2.4}\is{creole genesis}
What is evident from the discussions presented in \sectref{2.2.1} to \sectref{2.2.3} is that the typological classification of creoles is a major long-standing issue for \ili{Creole} linguists, and though several theories of \isi{genesis} have been proposed, it seems that one agreed upon theory of origin might never be found. Notwithstanding, all theorists, including those discussed in the previous sections, agree that in most cases the majority of the lexical entries in ``creoles'' are derived from their lexifier \isi{languages}, more specifically from 17\textsuperscript{th} century regional \isi{dialects} of their European lexifier \isi{languages} \citep{Chaudenson92, Chaudenson01, Mufwene96, Mufwene08, Lefebvre04, McWhorter05}. Of these theorists, two seminal works from within the same theoretical school of thought, i.e. the \isi{Superstratist} camp, provided me with varying theories regarding the nature of this \isi{Superstrate} \isi{lexical influence}. These were Mufwene's (\citeyear{Mufwene01, Mufwene08}) `All-the-\isi{dialects} view', referred to hereafter as the \isi{pan-dialectal account}, and  Chaudenson's (\citeyear{Chaudenson79, Chaudenson92, Chaudenson01}) `Single-dialect view', hereafter the \isi{mono-dialectal account}.

\subsubsection {The mono-dialectal account}\label{2.2.4.1}
 Chaudenson's (\citeyear{Chaudenson79}) work, as mentioned in \sectref{2.2.1}, addressed \ili{French} creoles, specifically those spoken in and around the Indian Ocean. He was able to show that the \isi{linguistic features} noticed in these creoles could be traced back to a particular 17\textsuperscript{th} to 18\textsuperscript{th} century \ili{French} dialect spoken in Normandy (Western France). Chaudenson did a detailed comparison of the structural correspondences between various \ili{French} \ili{Creole} reflexes and their \ili{French} \isi{dialectal etyma}. The assertion that \ili{French} creoles were genetically derived from 17\textsuperscript{th} to 18\textsuperscript{th} century \ili{French} was asserted from within the early works of \citet{Hall53}, and \citet{Goodman64}, whereas  \citet{Chaudenson79} was the first to pinpoint a precise dialect of \ili{French} from which this influence could be said to have originated.

\subsubsection {The pan-dialectal account}\label{2.2.4.2}
Mufwene's (\citeyear{Mufwene01}) \isi{pan-dialectal account} described a creole's \isi{lexical influence} as being made up of ``... composite varieties from among diverse \isi{dialects} of the same language ...'' (p.~3). This account provided a counter thesis to Chaudenson's account of \isi{Superstrate} influence. According to \citet{Mufwene08}, ``... the target for those who made the creoles consisted of several non-standard varieties [of the 17\textsuperscript{th} and 18\textsuperscript{th} century European lexifier \isi{languages} that were] competing with each other...'' (p.~21). These \isi{dialects} developed into a new colonial dialect that preserved at its core a mixture of common features found across the diverse \isi{dialects} of the \isi{lexifier language}. These competing features were thrown into a pan-dialectal ``feature pool'', thereby becoming accessible for selection by colonial speech communities \citep{Mufwene08, Mufwene01}. Survival of these features was ensured if they were more frequent, more salient, and/or more transparent than other alternatives. Survival was also guaranteed if a competing feature appeared in one of the other \isi{languages} present during the \isi{creole formation period}, for example one or more of the West African \isi{languages} \citep{Mufwene01}.

\subsubsection {The {pan-dialectal account} vs. the mono-dialectal Account}\label{2.2.4.3}\is{pan-dialectal account}
There is a possibility that Chaudenson is right as it relates to \ili{French}-\isi{lexicon} creole \isi{languages} and Mufwene is equally right about \ili{English}-\isi{lexicon} creole \isi{languages}. However, there is also the possibility that a Chaudenson-type approach might better account for the nature of the \ili{English} input in \ili{English}-\isi{lexicon} creoles. Since this is the case one of the questions that this research attempted to answer is which of the two accounts, i.e. Chaudenson's Mono-dialectal account or Mufwene's \isi{pan-dialectal account}, better explains the nature and origin of the lexical and phonetic input in \ili{Sranan}. To this end, this work refrains from focusing on the debate surrounding when, or how, what we know as modern \ili{Sranan} was formed. Instead, using some of the oldest and most available secondary \ili{Sranan} data, an attempt was made to reconstruct and trace the 17\textsuperscript{th} century putative \ili{English} input that would have influenced the Surinamese proto-language that soon developed into Proto-\ili{Sranan} and the two other \textsc{ssa} creoles, i.e. Proto-\ili{Saramaccan} and Proto-\ili{Aukan}. In so doing an attempt was made to investigate whether there were multiple dialectal inputs from all over \isi{England} or a single \isi{dialectal input} from within a specific region.
%Section2.2 ends %%%%%%%%%%%%%%%%%%%%%%%%%%%%%%%%%%%%%%%%%%%%%%

%Section2.3 starts %%%%%%%%%%%%%%%%%%%%%%%%%%%%%%%%%%%%%%%%%%%%%%
\subsection{The search for Sranan's English dialectal influence}
One seminal work, which addressed the \isi{Superstrate} origins of the Surinamese \ili{English} creoles, is Smith's (\citeyear{Smith87}), \textit{The \isi{genesis} of the creole \isi{languages} of Surinam}. In this work Smith used Historical Phonology and attempted to reconstruct the linguistic shape and origin of \textsc{ssa}'s 17\textsuperscript{th} century European input.  \citet[6]{Smith87}, utilising knowledge of  ``... seventeenth century \ili{English} and \ili{Portuguese}...'' alongside 18\textsuperscript{th} and 19\textsuperscript{th} century \textsc{ssa} data sources, attempted to find and discuss the correspondences between them. Smith's (\citeyear{Smith87}) major concern was the origin and ``linguistic interrelationships'' between \textsc{ssa}'s lexical and phonological influences, i.e. \ili{English}, \ili{Dutch} and \ili{Portuguese}.

Smith's (\citeyear{Smith87}) work, like Mufwene's (\citeyear{Mufwene01}), points to 17\textsuperscript{th} century regional \isi{dialects} of \ili{English} as the input for \ili{English} \isi{Creoles}. However, similar to \citet{Chaudenson79}, \citet{Smith87} was more explicit about the nature of \textsc{ssa}'s \ili{English} input. He proposed `Standard Early Modern London \ili{English}', with some input from regional non-standard \isi{dialects} around London, as the main input for \textsc{ssa} \citep{Smith08, Smith87}. Given Smith's proposed ``London \ili{English}'' account of origin, alongside determining whether a \isi{mono-dialectal account} or a \isi{pan-dialectal account} of origin can best explain \ili{Sranan}'s \ili{English} \isi{Superstrate} origin, one of the other major issues that this research also addressed was whether Smith's proposal is trustworthy. What this meant was that, in trying to determine whether the \ili{English} dialectal influence in \ili{Sranan} was from a composite of \isi{dialects} \citep{Mufwene08, Mufwene01} from all over \isi{England}, or a single \isi{regional dialect}, there was also an attempt to determine whether Smith's proposed ``London-\ili{English}'' was the only source of \ili{Sranan}'s \ili{English} \isi{linguistic influence}. In doing so, Smith's (\citeyear{Smith87}) work was taken as a stepping stone for this current research. However, whereas Smith's (\citeyear{Smith87}) work focussed on determining in a more general way the European origins (from \ili{Portuguese}, \ili{Dutch} and \ili{English}) of \ili{Sranan}'s lexical and phonetic influences, this current research focussed solely on the \ili{English} dialectal influence(s) from within \isi{England} in a bid to find the dialect or \isi{dialects} from within \isi{England} that were of \isi{linguistic influence}.

In a perfect world where things are as plain as ``black and white'', and shades of  ``grey'' are non-existent, an agreed upon definition of ``creole'' exists and the answers to all questions surrounding ``creole'' would have already been found. Sadly, this is not that perfect world; however, whatever the term ``creole'', in reference to language, means to the various researchers in the field of \ili{Creole} Linguistics, it is generally accepted that \ili{Sranan} is a creole language. This being the case, in taking Smith's (\citeyear{Smith87}) work as a stepping stone (see \chapref{ch:3}), a tripartite methodological model was created to determine the \isi{dialectal origin}(s), in \isi{England}, of those lexical and phonetic influences that needed to be present during the formation of this creole language. The components of this methodological model are discussed in the following chapter. This model consists of statistical analysis, Dialect Geography techniques and analysis of 17\textsuperscript{th} century \isi{England} migration history. As will become evident in the following chapters, the rationale behind using this methodological model is that it allowed for triangulation of results from three diverse types of analyses. Such corroboration of the results across all three types of analyses meant a higher degree of trustworthiness. This model and the data that were used in this research are discussed in greater detail in the following chapter.
%Section2.3 ends %%%%%%%%%%%%%%%%%%%%%%%%%%%%%%%%%%%%%%%%%%%%%%
 
