\chapter{A tale of two dialect inputs}\label{ch:7}

\section{Review}\label{7.1}
I had planned to, from the onset of this research, test the viability of a pan-dialectal versus \isi{mono-dialectal account} of origin for the \ili{English} dialectal influence in \ili{Sranan}. I had wanted to see whether the influence originated from all over \isi{England} or whether it originated from Smith's  proposed southeast \isi{England} \citep{Smith87, Smith08}. What I found, however, was more complex. The influence was neither from all over \isi{England}, nor from a single dialect region, nor was it even from southeast \isi{England} alone. Instead, the origin of the \ili{English} lexico-phonetic influence in \ili{Sranan} was from west-southwest \isi{England} and southeast \isi{England}. I arrived at these results via the use of a tripartite methodological tool consisting of a statistical component, a \isi{dialect geography} component and a historical component.

The results of the statistical analysis of the data provided statistically significant evidence for seven \isi{potential input} sources of origin for the lexical and phonetic influence in \ili{Sranan}. These seven lects are Blagdon, Wedmore, Horsington, Whitwell, Doddinghurst, Canewdon and East Mersea (see \chapref{ch:4}). The results of the \isi{linguistic feature mapping} corroborated the results of the statistical analysis; albeit presenting only a dense concentration of input forms in a region I dubbed the \textsc{wsw4} (see \chapref{ch:5}). This region, which is inclusive of localities within the counties of Somersetshire, Monmouthshire, Wiltshire and Gloucestershire, has at its core, a composite lect of origin that included three of the seven lects identified by the statistical component of analysis. These three lects are Blagdon, Wedmore and Horsington.

The Linguistic feature mapping highlighted neither Whitwell, which is situated in the south of \isi{England}, nor Canewdon, Doddinghurst and East Mersea, which are situated in the east and East Anglian county of Essex. However, the historical component of analysis, presented me with confirmation of not only the presence of a majority of 17\textsuperscript{th} century migrants from the \textsc{wsw4} region, but also migrants from the east and East Anglia region, alongside those counties immediately surrounding the two regions (see Map \ref{Map6.1} in \chapref{ch:6}). This third level of analysis, i.e. checking the 17\textsuperscript{th} century \ili{English} migration history from \isi{England}, triangulated the results of the statistical analysis and \isi{linguistic feature mapping} by presenting a number of facts. These are as follows:

\begin{enumerate}
\item{Prior to and during the period in which \isi{Suriname} was settled by the \ili{English} and subsequently lost to the \ili{Dutch}, i.e. 1630--1650 and 1650--1667, the \textsc{wsw4} region contributed the highest percentage of indentured servants to the \ili{English} colonies, from which \isi{Suriname} was settled (see \chapref{ch:6}).}
\item{East and East Anglian \isi{England} migrants were said to have migrated to the \ili{English} colonies in large numbers in the earlier period, i.e. from the 1630s to the 1640s. However, neither in the 1630s--1640s period, nor in the later 1650--1667 period did their numbers seem to outweigh those of the \textsc{wsw4} migrants going to \isi{Barbados} and St. Kitts, these being the two colonies from which \isi{Suriname} was settled.}
\item{Scotsmen and Irishmen were not present in the \ili{English} American colonies in any significant numbers during the period in which \isi{Suriname} was an \ili{English} colony. This is because migration from Ireland and Scotland did not seem to occur to any large degree, until 1716 for the Scots (see \citealt{Dobson05}) and between 1718 and 1785 for the \ili{Irish} \citep{Griffin01}.}
\end{enumerate}

These results indicated to me that \citet{Smith87, Smith08} was not wrong in his southeast \isi{origin hypothesis}; however, he may have neglected a significant west-southwest \isi{England} source. As stated in \chapref{ch:6}, none of the three components of analysis, could, by itself, account for all relevant \ili{English} linguistic and human presence in \isi{Suriname}. The combined results, however, painted a clearer picture, though it did so without telling me the tale of how this influence came to be. In this chapter, an attempt is made to tell, based on the combined results of the three components of analysis, the ``full'' tale of the Sranan45's 17\textsuperscript{th} century lexico-phonetic influence from \isi{England}.

\section{The tale being told}\label{7.2}
How do we explain what the results of \isi{statistics}, \isi{linguistic feature mapping} and history allude to? What did the east and East Anglian \isi{England} and the \textsc{wsw4} region of \isi{England} have in common, linguistically? Writing this tale involved my attempting to answer two major questions, which consisted of smaller, more specific parts. These questions were as follows:

\begin{enumerate}
\item{How did it happen? Did individual lexical items enter proto-\ili{Sranan} (the original version of the \ili{English} creole that was being created) via the mouths of individual native \ili{English} speakers using the own regional \isi{dialects}? Or, was there a levelled colonial \isi{Barbados}/\isi{Suriname} variety, which provided a single set of inputs, or was the source a combination
of the two?}
\item{Where did it happen? Where was the colonial \ili{English} input form coming from? Was it \isi{Barbados}/St. Kitts or did it develop in \isi{Suriname}?}
\end{enumerate}

\subsection{How and where did it happen?}\label{7.2.1}
In order to answer the two questions presented above, there was a need to put into context what the combined results of the tripartite system of analysis were indicating. This involved determining the nature of the linguistic relatedness between the input forms from the \textsc{wsw4} and southeast regions, respectively. It was theorized, based on the results of the three components of analysis, that I was dealing with a levelled and/or koineised \textsc{wsw4}$\sim$east and East Anglia composite lect of origin. To determine the plausibility of this theory, the degree of linguistic relatedness across the two regions was accessed (see \tabref{Table 7.1}).


\begin{table}
\begin{tabularx}{\textwidth}{lrrQ}
\lsptoprule 
\multicolumn{4}{c}{Lexico-phonetic shapes}\\\cmidrule(lr){1-4}
Region & \multicolumn{1}{c}{Items} & \multicolumn{1}{c}{\%} & Examples\\\midrule
Peculiar to \textsc{wsw4} & 15 & 33.3 & \textit{arse, ask, broth, corndoor, ear, eye, hearmaster, more, more, mouthstear, wear, woman}\\
Peculiar to East\slash East Anglian & 6 & 13.3 & \textit{brother, care, fingerfour, hurt, iron}\\
Shared across both & 24 & 53.3 & \textit{burn, cold, curse, fire, first, gold, gutter, hand, harehead, help, herring, hog, horse, hot, house, hungryold, remember, teeth, turn, work \textup{(n)}, work \textup{(v)}, yesterday}\\
\lspbottomrule
% % % % %  {15 {items with lexico-} \\ {phonetic shapes} \\ {peculiar to} \\ {the \emph{wsw4} region} \\ \\ (33.3\% of input forms)} &  {6 {items with lexico-} \\ {phonetic shapes} \\ {peculiar to the east and} \\ {east anglian region} \\ (13.3\% of input forms)}\\
% % % % %  {\textit{arse, ask, broth, corndoor, ear, eye, hearmaster, more, more, mouthstear, wear, woman,}} &  {\emph{brother, care, fingerfour, hurt, iron,}}\\
% % % % % \midrule
% % % % % \multicolumn{2} {c}{ {24 {items with lexico-phonetic shapes} \\ {shared across the two regions}
% % % % %  \\ (53.3\% of input forms) }}\\ 
% % % % % \multicolumn{2} {c}{{\emph{burn, cold, curse, fire, first, gold, gutter, hand, harehead, help, herring, hog, horse, hot, house, hungryold, remember, teeth, turn, work \textrm{(n)}, work \textrm{(v)}, yesterday}}}\\ 
\end{tabularx}
\caption{Distribution of shared and regionally peculiar \textsc{sed45}  input across the \textsc{wsw4} and east and East Anglia regions}
\label{Table 7.1}
\end{table}

The data presented in \tabref{Table 7.1} highlight the following facts: 

\begin{enumerate}
\item{The Sranan45 input originated from a \textsc{wsw4}$\sim$east and East Anglia composite lect of origin, with the east and East Anglian influence possibly being Smith's proposed Early Modern (London) \ili{English} \citep{Smith87, Smith08}. I arrived at this conclusion based on the fact that there is a high degree of linguistic relatedness across the two regions. Both regions had 53.3\% (\LSfrac{24/45}) of the same lexico-phonetic forms needed for the Sranan45 input. Also, when I added to this number the 33.3\% (\LSfrac{15/45}) and 13.3\% (\LSfrac{6/45}) of the input forms that were peculiar to the \textsc{wsw4} and east and East Anglia regions respectively, I secured all forty-five \textsc{sed} input forms for the Sranan45 reflexes.}
\item{The \textsc{wsw4} region provided 20\% (\LSfrac{9/45}) more exclusive \isi{regional input} forms, i.e. 33.3\% (\LSfrac{15/45}) than did the east and East Anglia region which provided 13.3\% (\LSfrac{6/45}).}
\item{The exclusive input forms provided by the east and East Anglian region were r-less and h-full and dental fricative-less, and the exclusive input forms provided by the \textsc{wsw4} region were both r-full and h-full, dental fricative-less and exhibited word-initial Palatals and consonant cluster reversal (See \chapref{ch:4} for discussion of these features).}
\end{enumerate}

These results indicated to me that my theory concerning the nature of the input was plausible; i.e. that the input was some \textsc{wsw4}$\sim$east and East Anglia composite lect. This meant that the plot for the tale of \ili{English} input for the Sranan45 was one that involved a number of possibilities. The first is that there was koineization in the colonies, i.e. either in \isi{Barbados} or in \isi{Suriname}, involving Smith's southeastern \isi{England}, Early Modern \ili{English} \citep{Smith87, Smith08} and \textsc{wsw4} \isi{England} \isi{linguistic features}. The second is that there were two linguistic codes existing alongside each other, in \isi{Barbados}/\isi{Suriname}, with the \textsc{wsw4} migrants, due to sheer numbers providing more of the \isi{linguistic influence} than the Easterners. The third possibility is that the \isi{Whites} all knew, had transported to the colonies and were using, features of the 1650s Early Modern \ili{English} standard that had begun to emerge in \isi{England} \citep{Smith87, Smith08}. They would do so while mixing in peculiar features from their respective \emph{wsw} and east and East Anglia \isi{dialects}.

\subsubsection{Working out the plot}\label{7.2.1.1}
I eventually arrived at the conclusion that the plot for the tale of input was one which, based on what the data presented in \tabref{Table 7.1} was telling me, included all three of the above-mentioned possibilities. I arrived at this conclusion for the reasons discussed hereafter.

Regional dialect levelling is not wholly dependent on the physical interaction of speech communities alone. One must consider the influence of the social element. This social element was in the form of an emerging Standard \ili{English} derived from southeastern \isi{England}, which was the dialectal \isi{lingua} franca of most persons in \isi{England} (see \citealt{Smith87, Smith08}). This explained the 53.3\% (\LSfrac{24/45}) shared items between the \emph{wsw} and east and East Anglia regions. It did not, however, explain the existence of the exclusive forms from both regions. This meant that the spread, in \isi{England}, of the ``Standard \ili{English}'' forms was possibly cut short due to the years of intense out-migration; however, in the colonies, there was the development of what \citet{Siegel85} referred to as an ``Immigrant \ili{Koine}'', i.e. the end result of koineization of different but related \isi{dialects} following a period of colonization (see \citealt{Siegel04}). Immigrant koines

\begin{quote}
... result from contact between regional \isi{dialects}; however, the contact takes place not in the region where the \isi{dialects} originate, but in another location where large numbers of speakers of different regional \isi{dialects} have migrated. Furthermore, it often becomes the primary language of the immigrant community and eventually supersedes the contributing \isi{dialects} \citep{Siegel85}.
\end{quote}

Like dialect levelling, koineisation involves, at first, the mixing of features from the different \isi{dialects} in contact. However, levelling is characterised by ``... the survival of one variant ... [such as r-full words alone] ... from a pool of competing ... [r-full and r-less] ... variants ...'' \citep[91]{Tuten03}. This is because ``dialect levelling is the process that reduces language variation'' \citep[313]{Hinskens09}. With, koineization, however, even though there is a mixing of the features of the varieties in contact; ``... the outcome of such convergence ... [of linguistic \isi{dialectal input}] ... is by no means complete uniformity ...'' \citep[193]{Bynon83}. In fact, as seen with the 33.3\% (\LSfrac{15/45}) peculiar items from the \textsc{wsw4} and the 13.3\% (\LSfrac{6/45}) for the east and East Anglian region, regular (shared) and exceptional (residual dialectal) forms existed side-by-side in the Sranan45 colonial \ili{English} input.

I accounted, in my mind, for how the input was created, but I was having a hard time determining whether the \ili{English} colonial input was formed in \isi{Barbados} or in \isi{Suriname}. I was unsure as to how to approach this portion of the plot concerning the tale of origin of the \ili{English} input in \ili{Sranan}. My belief is that the creation of the \ili{Sranan} input began in \isi{Barbados}, i.e. the immigrant koineization process and was continuing in \isi{Suriname} until this process was cut short because of the cession of \isi{Suriname} to the \ili{Dutch}. The only possible means by which to test this theory would be to take the \ili{English} creole spoken in \isi{Barbados} (hereafter \ili{Bajan}) and some of the lesser varieties of \ili{English}, which were spread from \isi{Barbados} to the rest of the Anglophone Americas, and compare their features.

Though not the focus of this study, an attempt at a phonological comparison was made between \ili{Sranan}, \ili{Bajan} and some of the \ili{English} varieties spoken in the Leeward Islands (see \citealt{Aceto10, Blake04, Aceto04, Williams03}), using two of the dominant phonetic features that presented themselves in the data (see \chapref{ch:3}). These are the $\pm$h and $\pm$\textsc{pvr} features.

The rationale was that if the koineization process was cut short in \isi{Suriname}, due to the colony's cession to the \ili{Dutch}, then, though I should see some similarities between say \ili{Bajan} and \ili{Sranan}, I should not see, for example, the rhotic and non-rhotic variation noticed with \ili{Sranan}. \tabref{Table 7.2} below presents a presentation of what I found out.

\begin{table}
\begin{tabular}{lccc}
\lsptoprule 
{Features} & {Bajan} & {Sranan} & {Leeward Islands}\\
\midrule
{$\pm$\textsc{pvr}} & \emph{+\textsc{pvr}} & \emph{$\pm$\textsc{pvr}} & \emph{\textminus\textsc{pvr}}\\
{$\pm$h} & \emph{+h} & \emph{+h} & \emph{+h}\\
\lspbottomrule 
\end{tabular}
\caption{Linguistic feature comparison: Sranan, Bajan and Leeward Island Varieties (Antigua, St. Kitts, Nevis, Montserrat, Anguilla, Barbuda)}
\label{Table 7.2}
\end{table}

The data presented in \tabref{Table 7.2} illustrate a phonological continuum. At the midpoint of the continuum was \ili{Sranan}, whose koineization/\isi{levelling process} was cut short, as illustrated by the presence of variability in \isi{rhoticity}. To the right and left of \ili{Sranan} are \ili{Bajan} and the Leeward Islands varieties, with no variation in \isi{rhoticity}, which means that the koineization/\isi{levelling process} was completed in favour of Rhoticity for \ili{Bajan} and non-\isi{rhoticity} for the Leeward Islands varieties. This is illustrated by the fact that \ili{Bajan} is rhotic and the Leeward Islands varieties are non-rhotic. For me this meant that there was a possibility that any \ili{Bajan} immigrant \isi{koine} was not yet fully created by the time migrants left \isi{Barbados} for \isi{Suriname}. Based on the above discussion the following tale of origin can be proposed for the \isi{lexico-phonetic input} from \isi{England} into \ili{Sranan}. In pursuit of a better life and for fear of incarceration due to political affiliation during the Great Civil War, two great 17\textsuperscript{th} century exoduses occurred from \isi{England}. The destination of the migrants from \isi{England} was the \ili{English}-owned American colonies, such as \isi{Barbados} and St. Kitts. These colonies represented a chance for a new life away from the social and political troubles they faced in \isi{England}.

Migrants came in great numbers, specifically from within and around the \textsc{wsw4} and the east and East Anglian regions of \isi{England}. These migrants shared common \isi{linguistic features}, but they also used linguistic forms that were peculiar to their respective regions. In \isi{Barbados}, their mix of shared and peculiar linguistic forms became the colonial target language for enslaved Africans who soon began to enter the colonies. Some of the migrants now turned colonists, and the enslaved Africans, eventually left or were taken from \isi{Barbados} to settle in other England-owned colonies such as \isi{Suriname} and the islands of the Eastern \isi{Caribbean}. For those who went to \isi{Suriname}, the variability found within the colonial \ili{English} that they spoke remained with the enslaved Africans who remained in the country after its cession to the \ili{Dutch}.

For those migrants and enslaved Africans who went to the Leeward Islands, they eventually levelled out this variability in favour of a non-rhotic accent but for those who stayed in \isi{Barbados} they levelled out the variability in favour of a rhotic one. \ili{Sranan} therefore represents a linguistic fossil of the early colonial \ili{English} that contributed to its development. It was \isi{Colonial} \ili{English} that survived in an \ili{English} creole, within a social and linguistic ecology that was devoid of continued \isi{Colonial} \ili{English} \isi{linguistic influence}, from outside of \isi{Suriname}, after 1667.

\subsection{The way forward}\label{7.2.2}
The cutting edge of \isi{linguistic research} is to be found in the use of methodological tools such as the one used in this research, to reconstruct language history. My work could be taken as a spin-off of those works that seek to reconstruct, via mathematical procedures the spread of human \isi{languages}. One such contemporary work is that of \citet[348]{Atkinson11}, who, via statistical analysis of the ``... phoneme counts derived from the Worlds Atlas of Language Structures ... [database]'' was allegedly able to show that ``... global phonemic diversity was shaped by a serial founder effect during the human expansion from \isi{Africa} ... ''.

Unlike \citet{Atkinson11}, my research did not concentrate on attempting to reconstruct the spread of linguistic genetic features around the world. Instead, via a tripartite tool, which also used \isi{dialect geography} and \isi{historical data} as a means of triangulating the results of the statistical analysis, I attempted a reconstruction of the spread of \ili{English} lexico-phonetic genetic features from \ili{English} into a creole language spoken in \isi{Suriname}. The use of this methodological tool can bear fruit if applied to all Atlantic \ili{English} creoles and also the relatively understudied `lesser-known varieties of \ili{English}' (see \citealt{Aceto04, Williams03, Trudgill02}). This is because it allows us to trace the relationships across these creole \isi{languages} and also to determine the origin(s) of the original lexico-phonetic stock from \isi{England} for example.
The use of the tool is, of course, dependent on our ability to secure the most archaic forms of our creole of interest, especially in cases where decreolization has resulted in certain \ili{English} creole \isi{languages} becoming more like their lexifier \isi{languages}. The tool is also applicable to the stock of African \isi{lexicon} in relation to creole forms coming from various African \isi{languages}. Imagine being able to determine, for example, the precise Eastern \ili{Ijo} dialectal region from which the \ili{Ijo} words in \ili{Berbice} \ili{Dutch} creole originated. The future path of comparative \isi{linguistic research} is clear; it is to be found in the use of mathematically-based methodologies such as the one that was fashioned and used to undertake this research.
